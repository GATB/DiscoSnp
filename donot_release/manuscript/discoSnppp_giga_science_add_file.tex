%% BioMed_Central_Tex_Template_v1.06
%%                                      %
%  bmc_article.tex            ver: 1.06 %
%                                       %

%%IMPORTANT: do not delete the first line of this template
%%It must be present to enable the BMC Submission system to
%%recognise this template!!

%%%%%%%%%%%%%%%%%%%%%%%%%%%%%%%%%%%%%%%%%
%%                                     %%
%%  LaTeX template for BioMed Central  %%
%%     journal article submissions     %%
%%                                     %%
%%          <8 June 2012>              %%
%%                                     %%
%%                                     %%
%%%%%%%%%%%%%%%%%%%%%%%%%%%%%%%%%%%%%%%%%


%%%%%%%%%%%%%%%%%%%%%%%%%%%%%%%%%%%%%%%%%%%%%%%%%%%%%%%%%%%%%%%%%%%%%
%%                                                                 %%
%% For instructions on how to fill out this Tex template           %%
%% document please refer to Readme.html and the instructions for   %%
%% authors page on the biomed central website                      %%
%% http://www.biomedcentral.com/info/authors/                      %%
%%                                                                 %%
%% Please do not use \input{...} to include other tex files.       %%
%% Submit your LaTeX manuscript as one .tex document.              %%
%%                                                                 %%
%% All additional figures and files should be attached             %%
%% separately and not embedded in the \TeX\ document itself.       %%
%%                                                                 %%
%% BioMed Central currently use the MikTex distribution of         %%
%% TeX for Windows) of TeX and LaTeX.  This is available from      %%
%% http://www.miktex.org                                           %%
%%                                                                 %%
%%%%%%%%%%%%%%%%%%%%%%%%%%%%%%%%%%%%%%%%%%%%%%%%%%%%%%%%%%%%%%%%%%%%%

%%% additional documentclass options:
%  [doublespacing]
%  [linenumbers]   - put the line numbers on margins

%%% loading packages, author definitions

%\documentclass[twocolumn]{bmcart}% uncomment this for twocolumn layout and comment line below
\documentclass{bmcart}

%%% Load packages
%\usepackage{amsthm,amsmath}
%\RequirePackage{natbib}
%\RequirePackage{hyperref}
\usepackage[utf8]{inputenc} %unicode support
%\usepackage[applemac]{inputenc} %applemac support if unicode package fails
%\usepackage[latin1]{inputenc} %UNIX support if unicode package fails


%%%%%%%%%%%%%%%%%%%%%%%%%%%%%%%%%%%%%%%%%%%%%%%%%
%%                                             %%
%%  If you wish to display your graphics for   %%
%%  your own use using includegraphic or       %%
%%  includegraphics, then comment out the      %%
%%  following two lines of code.               %%
%%  NB: These line *must* be included when     %%
%%  submitting to BMC.                         %%
%%  All figure files must be submitted as      %%
%%  separate graphics through the BMC          %%
%%  submission process, not included in the    %%
%%  submitted article.                         %%
%%                                             %%
%%%%%%%%%%%%%%%%%%%%%%%%%%%%%%%%%%%%%%%%%%%%%%%%%


% \def\includegraphic{}
% \def\includegraphics{}



%%% Put your definitions there:
\startlocaldefs

\usepackage{xspace}
\newcommand{\disco}{{\it DiscoSnp}\xspace}
\newcommand{\discopp}{{\it DiscoSnp++}\xspace}

\usepackage{graphicx}

\endlocaldefs


%%% Begin ...
\begin{document}

%%% Start of article front matter
\begin{frontmatter}

\begin{fmbox}
\dochead{Technical Note}

%%%%%%%%%%%%%%%%%%%%%%%%%%%%%%%%%%%%%%%%%%%%%%
%%                                          %%
%% Enter the title of your article here     %%
%%                                          %%
%%%%%%%%%%%%%%%%%%%%%%%%%%%%%%%%%%%%%%%%%%%%%%

\title{{\it DiscoSnp++}: Additional File}

%%%%%%%%%%%%%%%%%%%%%%%%%%%%%%%%%%%%%%%%%%%%%%
%%                                          %%
%% Enter the authors here                   %%
%%                                          %%
%% Specify information, if available,       %%
%% in the form:                             %%
%%   <key>={<id1>,<id2>}                    %%
%%   <key>=                                 %%
%% Comment or delete the keys which are     %%
%% not used. Repeat \author command as much %%
%% as required.                             %%
%%                                          %%
%%%%%%%%%%%%%%%%%%%%%%%%%%%%%%%%%%%%%%%%%%%%%%

\author[
   addressref={aff1},                   % id's of addresses, e.g. {aff1,aff2}
   corref={aff1},                       % id of corresponding address, if any
   % noteref={n1},                        % id's of article notes, if any
   email={pierre.peterlongo@inria.fr}   % email address
]{\inits{PP}\fnm{Pierre} \snm{Peterlongo}}
 \author[
    addressref={aff1},
    email={erwan.drezen@inria.fr}
 ]{\inits{ED}\fnm{Erwan} \snm{Drezen}}

%%%%%%%%%%%%%%%%%%%%%%%%%%%%%%%%%%%%%%%%%%%%%%
%%                                          %%
%% Enter the authors' addresses here        %%
%%                                          %%
%% Repeat \address commands as much as      %%
%% required.                                %%
%%                                          %%
%%%%%%%%%%%%%%%%%%%%%%%%%%%%%%%%%%%%%%%%%%%%%%

\address[id=aff1]{%                           % unique id
  \orgname{GenScale, INRIA Rennes Bretagne-Atlantique, IRISA}, % university, etc
  \street{Campus de Beaulieu},                     %
  %\postcode{}                                % post or zip code
  \city{Rennes},                              % city
  \cny{France}                                    % country
}
% \address[id=aff2]{%
%   \orgname{Marine Ecology Department, Institute of Marine Sciences Kiel},
%   \street{D\"{u}sternbrooker Weg 20},
%   \postcode{24105}
%   \city{Kiel},
%   \cny{Germany}
% }

%%%%%%%%%%%%%%%%%%%%%%%%%%%%%%%%%%%%%%%%%%%%%%
%%                                          %%
%% Enter short notes here                   %%
%%                                          %%
%% Short notes will be after addresses      %%
%% on first page.                           %%
%%                                          %%
%%%%%%%%%%%%%%%%%%%%%%%%%%%%%%%%%%%%%%%%%%%%%%

\begin{artnotes}
%\note{Sample of title note}     % note to the article
% \note[id=n1]{Equal contributor} % note, connected to author
\end{artnotes}

\end{fmbox}% comment this for two column layout

%%%%%%%%%%%%%%%%%%%%%%%%%%%%%%%%%%%%%%%%%%%%%%
%%                                          %%
%% The Abstract begins here                 %%
%%                                          %%
%% Please refer to the Instructions for     %%
%% authors on http://www.biomedcentral.com  %%
%% and include the section headings         %%
%% accordingly for your article type.       %%
%%                                          %%
%%%%%%%%%%%%%%%%%%%%%%%%%%%%%%%%%%%%%%%%%%%%%%

% \begin{abstractbox}
%
% \begin{abstract} % abstract
% \parttitle{First part title} %if any
%
% NGS data provides an unprecedented access to life mechanisms. In particular these data enable to detect polymorphisms such as SNPs and indels.
% These polymorphisms represent a fundamental source of information in agronomy, environment or medicine.  Thus detecting these polymorphisms is a routine task with NGS data. The main methods for their prediction usually need a reference genome. However, non-model organisms and highly divergent genomes such as in cancer studies  are more and more investigated.
%
% \parttitle{Second part title} %if any
% Text for this section.
% \end{abstract}
%
% %%%%%%%%%%%%%%%%%%%%%%%%%%%%%%%%%%%%%%%%%%%%%%
% %%                                          %%
% %% The keywords begin here                  %%
% %%                                          %%
% %% Put each keyword in separate \kwd{}.     %%
% %%                                          %%
% %%%%%%%%%%%%%%%%%%%%%%%%%%%%%%%%%%%%%%%%%%%%%%
%
% \begin{keyword}
% \kwd{sample}
% \kwd{article}
% \kwd{author}
% \end{keyword}
%
% % MSC classifications codes, if any
% %\begin{keyword}[class=AMS]
% %\kwd[Primary ]{}
% %\kwd{}
% %\kwd[; secondary ]{}
% %\end{keyword}

% \end{abstractbox}
%
%\end{fmbox}% uncomment this for twcolumn layout

\end{frontmatter}

%%%%%%%%%%%%%%%%%%%%%%%%%%%%%%%%%%%%%%%%%%%%%%
%%                                          %%
%% The Main Body begins here                %%
%%                                          %%
%% Please refer to the instructions for     %%
%% authors on:                              %%
%% http://www.biomedcentral.com/info/authors%%
%% and include the section headings         %%
%% accordingly for your article type.       %%
%%                                          %%
%% See the Results and Discussion section   %%
%% for details on how to create sub-sections%%
%%                                          %%
%% use \cite{...} to cite references        %%
%%  \cite{koon} and                         %%
%%  \cite{oreg,khar,zvai,xjon,schn,pond}    %%
%%  \nocite{smith,marg,hunn,advi,koha,mouse}%%
%%                                          %%
%%%%%%%%%%%%%%%%%%%%%%%%%%%%%%%%%%%%%%%%%%%%%%

%%%%%%%%%%%%%%%%%%%%%%%%% start of article main body
% <put your article body there>

%%%%%%%%%%%%%%%%
%% Background %%
%%

\section*{Data Simulation}

\subsection*{Simulating 2 human read sets}
TODO
\subsection*{Simulating 2 to $n$ E. Coli datasets}
We propose an experiment simulating more than two haploid bacterial individuals. For doing this, we created 30 copies (that we call individuals) of the E. coli K-12 MG1655 strain. We then simulated SNPs with a uniform distribution such that $\approx$4200 SNPs ($\approx$0.1\% of the genome length) are common to any pair of individuals, half this number is common to any trio of individuals, a third to any quadruplet, and so on. With this strategy, while considering all the 30 individuals together, 69 600 SNP sites were generated, covering $\approx$1.5\% of the genome. We also simulated INDELs following exactly the same process, with a ratio of one INDEL for ten SNPs. 

We simulated a 40x sequencing of each of the 30 individuals, with 100-bp reads and 0.1\% error rate. Thus, 1 855 870 reads were generated per read set.



\section*{Validation of predictions and precision/recall computations}
Both on real datasets or synthetic ones, predicted polymorphisms can be compared to a reference set. In all the performed tests, one dispose from a reference genome that is used only in the purpose of simulating and validating predictions.
Whatever the tested method, the predicted polymorphisms are validated using the following process:
\begin{itemize}
	\item All predicted sequences are fully mapped using Gassst~\cite{Rizk2010} on the reference genome with a least 90\% similarity.
	\item For SNPs:
	\begin{itemize}
		\item For each couple of sequences predicting one or more SNPs: if one or the two sequence(s) map the reference genome: for each predicted SNP of the mapped sequence(s), if its position matches exactly a position on which a SNP was simulated, then this SNP is considered as a True Positive (TP).
	\end{itemize}
	\item For indels:
	\begin{itemize}
		\item If one or the two sequence(s) of a predicted indel matches a simulated indel position, then such a prediction is considered as a True Positive (TP).
	\end{itemize}
	\item A predicted polymorphism that is not a True Positive is a False Positive (FP).
	\item A simulated polymorphism not mapped by a predicted polymorphism is a False Negative (FN).
\end{itemize}

The precision defined by $$precision=\frac{number TP}{number TP+ number FP}$$ and the recall is defined by $$recall=\frac{number TP}{number TP+ number FN}$$.

% if your bibliography is in bibtex format, use those commands:
\bibliographystyle{bmc-mathphys} % Style BST file
\bibliography{discoSnppp_giga_science}      % Bibliography file (usually '*.bib' )
\end{document}
