\documentclass[french,10pt]{article}
\usepackage[T1]{fontenc}
\usepackage[utf8]{inputenc}  
\usepackage{fullpage}
\usepackage{babel}
\usepackage{graphicx}
\usepackage{palatino}
\usepackage{verbatim,moreverb}
\usepackage{amsmath,bbold}
\usepackage{algorithm,algorithmic}


%%%%%%%%%%%%%

%\newlength{\lon}
\newcounter{quest}

\newcommand{\question}[1]{
 \addtocounter{quest}{1} \bigskip
%\settowidth{\lon}{\noindent {\bf Q \arabic{quest}.#1}}
\noindent {\bf Q \arabic{quest}.#1}\,\,  %\hskip+1cm %\hskip-\lon 
}


\newcommand{\qu}{\question{}}
\newcommand{\qb}{\question{$\mathbf{{}^{*}}$}}
\newcommand{\qbb}{\question{$\mathbf{{}^{**}}$}}

%%%%%%%%%%%%%

% \newcommand{\solu}[1]{\displaysol{#1}} % Avec solutions
\newcommand{\solu}[1]{}                % Sans solutions

\newcommand{\displaysol}[1]{
   \textbf{Sol :} 
  #1 
  \textbf{--Fin sol}
  }

\newcommand{\sol}[1]{\solu{~\\ #1}}
\newcommand{\soll}[1]{\solu{~\\ #1}}

\newcommand{\solution}[1]{\solu{#1}}

%%%%%%%%%%%%%

\newcommand{\ligne}{\vskip1cm
\centerline{\_\_\_\_\_\_\_\_\_\_\_\_\_\_\_\_\_\_\_\_\_\_\_\_\_\_\_\_\_\_\_\_\_\_\_\_}
\vskip1cm}
\newcommand{\fl}{\longrightarrow}
\newcommand{\eps}{\varepsilon}
\newcommand{\spa}{\phantom{1234}}
\usepackage{array}
\usepackage{url}
\usepackage{moreverb}
\usepackage{verbatim}
\usepackage[utf8]{inputenc}
\usepackage[T1]{fontenc}
\usepackage{xcolor}
\usepackage{xspace}
\usepackage{pifont}
\newcommand{\discopp}{{\it DiscoSnp++}\xspace}
\newcommand{\code}[1]{\begin{center} \textcolor{gray}{{\emph{#1}}} \end{center}} 
\newcommand{\sligne}{\begin{center} \begin{tabular}{c}~~~~~~~~~~~~~~~~~~~~~~~~~~~~~~~~~~~~~~\\ \hline\\\end{tabular}\end{center}}

\begin{document}
%%%%%%%%%%%%%%%%%%%%%%%%%%%%%%%%%%%%%%%%%%%%
\title{Formation Bioinformatique pour le traitement de données de séquençage (NGS)\\
TP \discopp}
\date{Mars 2015} 
\author{Pierre Peterlongo}
\maketitle
%%%%%%%%%%%%%%%%%%%%%%%%%%%%%%%%%%%%%%%%%%%%
\textbf{}
%%%%%%%%%%%%%%%%%%%%

%%%
\section*{Préambule} % (fold)
\label{sec:preambule}
Lors de ce TP nous débuterons par l'utilisation de \discopp intégralement en ligne de commande. Les lignes de commandes spécifiées dans cet énoncé sont formatée ainsi:
\code{command}

Toutes les questions ont un intérêt (bien sur). Cependant les question clefs de ce TP sont précédées du symbole '\ding{253}'.
% section preambule (end)

\section*{Téléchargement et installation de \discopp} % (fold)
\label{sec:telechargement_et_installation}
Nous allons débuter par télécharger et installer \discopp. Pour cela allez sur la page: \url{http://colibread.inria.fr/software/discosnp/} et télécharger la dernière version de \discopp. \\

L'archive ``DiscoSNP++-X.Y.Z-Source.tar.gz'' se trouve maintenant sur votre machine. Trouvez la puis décompressez la:


\code{tar -xvzf DiscoSNP++-X.Y.Z-Source.tar.gz}

Une fois dans le répertoire créé (\emph{``DiscoSNP++-X.Y.Z-Source''}), vous pouvez compiler les outils à l'aide de la commande suivante:

\code{./compile\_discoSnp++.sh}


% \begin{itemize}
%        \item Récupérons deux deux jeux de reads: \code{wget \url{http://igm.univ-mlv.fr/~peterlon/humch1_00096_reads.fasta.gz}\\wget \url{http://igm.univ-mlv.fr/~peterlon/humch1_00100_reads.fasta.gz}}
%        \item Récupérons un génome de référence et des données associées: \code{wget \url{http://igm.univ-mlv.fr/~peterlon/humch1_first_10M.fasta}\\wget \url{http://igm.univ-mlv.fr/~peterlon/ref_human}}
%        \item Récupérons des outils spécifiques à ce TP: \code{wget \url{http://igm.univ-mlv.fr/~peterlon/tools.zip}}
%        \item Préparons ces outils:
%        \code{unzip tools.zip\\cd tools/gassst\\./make \#make osx=1 si vous êtes sous mac\\cd ../..\\ln -s tools/* .}
% \end{itemize}





\section*{Premiers pas}
La commande principale est ``run\_discoSnp++.sh''. Nous pouvons l'appeler sans arguments ou avec des arguments. Commençons par lancer la commande sans arguments:
\code{./run\_discoSnp++.sh}
Dans ce cas, l'aide est affichée. Cette commande est équivalente à l'appel de ``run\_discoSnp++.sh'' avec l'option/l'argument ``-h'':
\code{./run\_discoSnp++.sh -h}



%\qu Lisez attentivement les options possibles. 

\qu \ding{253} Lisez le fichier README et lancez la première commande QUICK START indiquée.

\qu \ding{253} Quels fichiers ont été générés ? Indiquez pour chacun d'entre eux leur contenu et leur utilité. 

\qu \ding{253} Lancez la seconde commande indiquée dans le README. Quelle différence lors de l'appel de ``run\_discoSnp++.sh'' avec ou sans l'option -G ?

\section*{Application sur un fragment du génome humain en version ligne de commandes}

\subsection*{Pré-requis pour l'analyse des données humaines}
Nous allons créer et utiliser un répertoire spécifique pour cette partie du TP. Dans le répertoire (\emph{``DiscoSNP++-X.Y.Z''}) effectuez les commandes suivantes:
\code{mkdir tp\_disco\\cd tp\_disco}

\qu \ding{253} Comment appeler \discopp à partir d'un répertoire ne contenant pas le script ``run\_discoSnp++.sh'' ?\\

Nous allons utiliser des jeux de données et des outils spécifiques à ce TP. Voici les commandes à utiliser afin d'utiliser correctement ces ressources:
        \code{wget \url{http://igm.univ-mlv.fr/~peterlon/tp_installer.sh} \# 10 millions}
        \code{wget \url{http://www.irisa.fr/symbiose/people/ppeterlongo/tp_installer5M.sh} \# 5 millions}
        
        \code{chmod u+x tp\_installer.sh}
        \code{./tp\_installer.sh}
        
        
Les fichiers humch1\_00096\_reads.fasta.gz et humch1\_00100\_reads.fasta.gz sont deux jeux de reads simulés correspondant à deux individus humains issus du projet 1000 génomes.  Nous avons conservé les 10  premiers millions de nucléotides du génome humain hg19. Nous avons simulé les variants détectés par le projet 1000 génomes pour ces 2 individus. Ces deux jeux de reads correspondent à un séquençage Illumina à profondeur $40x$ pour chacun des 2 individus, avec des reads de taille 100 et un taux d'erreur uniforme de 0.1\%.



Nous avons également à notre disposition le fichier ``humch1\_first\_10M.fasta'' qui est notre génome de référence . Il s'agit du génome hg19 limité lui aussi au 10  premiers millions de nucléotides du chromosome 1.


\subsection*{Premiers tests}
\qu Que signifie la notion de SNP \emph{isolé} / SNP \emph{proches} ?

\qu Quelle option permet  de détecter ou non des SNPs dits \emph{proches} dans les reads de ces deux individus ? 

\qu \ding{253} Lancez \discopp sur les reads humains en ne recherchant que des SNP \emph{isolés}.

\qu \ding{253} Relancez \discopp en détectant jusqu'à 2 SNPs proches dans une même \emph{bubble} ?

\qu Ouvrez le fichier \emph{"coherent.fa"} ainsi généré à l'aide d'un éditeur de texte, puis recherchez des résultats de SNPs proches.

\qu Lors de votre dernière commande vous avez recalculé le graphe de de Bruijn des $k$-mers des deux jeux de reads (fichier .h5). Cela vous parait il judicieux ? Combien de temps cela a t'il pris ? Comment éviter ce calcul redondant ? 


\qu \ding{253} Jusqu'à présent \discopp n'a pas détecté d'indels. Pouvez vous expliquer pourquoi ? Indiquez puis exécutez la commande permettant de détecter des indels de taille $\leq$ 10.


\subsubsection*{Extensions}
\qu Jusqu'à présent \discopp n'a pas détecté les contigs auxquels appartiennent les variants détectés. Quelle option permet de détecter ces contigs ? 

\qu Lancez \discopp pour détecter 1/ jusqu'à 2 SNPs proches dans une même \emph{bubble} 2/ des indels de taille $\leq$ 10 3/ les contigs auxquels appartiennent les variants détectés.

\qu Si tout va bien nous avons lancé plusieurs fois \discopp avec succès. Comment retrouvez vous les résultats de ces différents appels ? 

\subsection*{Validation des résultats}
Sachant que nous avons nous même généré les données nous disposons d'une liste parfaite et exhaustive des variants qu'il faut retrouver. Ceci nous permet d'estimer la précision (proportion de mauvais variants prédits) et le recall (proportion de variants correctement détectés).

Nous vous proposons un outil \emph{validator.sh} permettant de calculer automatiquement des statistiques sur les prédictions \discopp. En lui indiquant le fichier \emph{coherent.fa} généré par \discopp, ce script génère :
\begin{itemize}
       \item Un fichier .log qui indique pour les SNPs et les Indels la précision et le recall. 
       \item Un fichier .png qui montre la précision et le recall en fonction du rang. 
       \item Un fichier \_stat.txt qui indique en particulier le rang moyen des vrais positifs (TP) et des faux positifs (FP) ainsi que la longueur des unitigs et contigs en fonction du statut TP ou FP.
       \item Un fichier \_stat.png qui montre des boxplots de la répartition des rangs, longueurs des unitigs ou contigs en fonction du statut des prédiction (TP ou FP).
\end{itemize}

\qu Retrouvez le fichier \emph{$\dots$coherent.fa} de votre dernier appel de \discopp. Il devrait s'appeler \emph{discoRes\_k\_31\_c\_4\_D\_10\_P\_2\_b\_0\_withlow\_coherent.fa}. Lancez la commande 
\code{./validator.sh discoRes\_k\_31\_c\_4\_D\_10\_P\_2\_b\_0\_withlow\_coherent.fa}

\qu Lisez le fichier \emph{discoRes\_k\_31\_c\_4\_D\_10\_P\_2\_b\_0\_withlow\_coherent.log}
\code{more discoRes\_k\_31\_c\_4\_D\_10\_P\_2\_b\_0\_withlow\_coherent.log}
Comment interprétez-vous les informations qui y sont présentées ?

\qu Même question pour le fichier \emph{discoRes\_k\_31\_c\_4\_D\_10\_P\_2\_b\_0\_withlow\_coherent\_stat.txt} et la figure \emph{discoRes\_k\_31\_c\_4\_D\_10\_P\_2\_b\_0\_withlow\_coherent\_stat.png}.


%
% \begin{itemize}
%        \item tests $k$
%        \item tests $c$
%        \item compare $D,P$
%        \item test $C$ 100
% \end{itemize}

\subsubsection*{Pour aller plus loin} % (fold)
\label{ssub:pour_aller_plus_loin}
Deux paramètres sont critiques lors de l'appel à \discopp: $k$: la longueur des $k$-mers utilisés dans le graphe de de-Bruijn et $c$ le seuil de comptage pour conserver un $k$-mer (tous les $k$-mers vus moins de $c$ fois dans les jeux de données ne sont pas utilisés). Notez que les valeurs utilisées par défaut ont été déterminées empiriquement et ne conviendront pas à toutes les instances. 

\qu Quelle est la valeur de $c$ par défaut ?

\qu Lancez \discopp sans autre option que $c=2$, puis $c=4$ puis $c=10$. Comparez les résultats obtenus en utilisant \emph{validator.sh}. Qu'en concluez vous ?
%for c in 2 4 10; do ../../../run_discoSnp++.sh -r "humch1_00096_reads.fasta.gz humch1_00100_reads.fasta.gz" -c $c; done
%for c in 2 4 10; do ./validator.sh discoRes_k_31_c_$c\_D_0_P_1_b_0_withlow_coherent.fa; done


\qu Quelle est la valeur de $k$ par défaut ?

\qu Lancez \discopp sans autre option que $k=11$, puis $k=31$, puis $k=61$. Comparez les résultats obtenus en utilisant \emph{validator.sh}. Qu'en concluez vous ?
%for k in 11 31 61; do ../../../run_discoSnp++.sh -r "humch1_00096_reads.fasta.gz humch1_00100_reads.fasta.gz" -k $k -g; done

\qu À l'inverse de l'option -c, il existe également un paramètre -C permettant de limiter le nombre maximal d'apparitions d'un $k$-mer dans chaque jeu de données. Quelle est l'utilité d'un tel paramètre ? 

\qu Comment utiliser des données pairées ? Quel est l'impact sur le déroulement de discoSnp++ ?

\qu Lancez \discopp en limitant le nombre d'apparitions des $k$-mer à 100 (sans autre option). Comparez avec \emph{validator.sh} les résultats sans l'option -C. Qu'en concluez-vous ?

\qu Lancez \discopp en détectant jusqu'à 40 SNPs proches dans une même \emph{bubble}. Que se passe t'il ? 
% subsubsection pour_aller_plus_loin (end)

\section*{Utilisation de \discopp sous Galaxy}
Vous trouverez à cette adresse: \url{http://galaxy.genouest.org/} une instance galaxy intégrant les outils du projet Colib'read et en particulier \discopp.

Vous pourrez accéder à cette instance via les logins/mots de passe qui vous seront fournis lors du TP. 

\subsection*{Interface galaxy}
\qu \ding{253} Comment uploader vos propres données ? Comment utiliser les données déjà préparées pour ce TP ?

\qu \ding{253} Où trouver les outils Colib'read et plus particulièrement \discopp ?

\qu Une fois sur la page de lancement de \discopp retrouvez les options que nous avons déjà découvertes. Certaines options ne sont pas accessibles depuis l'interface Galaxy. Est-ce limitant pour vous ?

\subsection*{Test sur plus de deux individus}

\qu \ding{253} Lancez sous galaxy \discopp sur un sous-ensemble de quelques individus \emph{E. Coli}, avec mapping sur génome de référence. Note: les données \emph{Coli} sont disponibles dans les librairies de données pour ce TP. 

\qu Quand et où retrouvez vous vos résultats ?

\qu Comment retrouvez vous l'information liée à chacun des individus que vous avez utilisés ? 


\subsection*{Test sur un unique haploïde}
% Les données proposées pour ce TP contiennent
% \begin{itemize}
%        \item Les données humaines que nous venons de tester
%        \item Des données issues de 10 individus E. Coli. Dans lesquelles des variants on été insérés de manière à ce que : la moitiée des variants soit partagé par toute paire d'individus, 1/3 par 3, ...
% \end{itemize}
\qu \ding{253} Lancez \discopp sur un unique jeu de reads \emph{E. Coli} (coli\_muted\_n\_30\_genome\_0\_reads.fasta par exemple).

\qu Que pouvez vous dire des résultats ? Aurions nous conclus la même chose sur les données humains (jeux humch1\_00096\_reads.fasta.gz seul par exemple) ?



\end{document}

